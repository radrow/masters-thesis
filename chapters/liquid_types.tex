\chapter{Liquid types}
\label{liquid_types}

\epigraph{The purpose of a programming language is to make it easier to
  express the things that do make sense while making it harder or impossible to
  express things that do not make sense.}{Ulf Norell}

\section{Motivation}

Smart contracts are gaining more and more popularity these days and are being
used to manipulate assets, whose total value exceeds many other digital
entities. With the increasing importance comes increased demand on the quality
and security. Due to this trend and the limitations of human perception, there
comes a need of using computers to accomplish these necessities.

Static semantics have been used successfully in keeping programs safe in terms
of the execution stability and following the intentions of the programmer
\cite{Lagouvardos:2020:PSM, Coblenz:2020:OTA,8445052,Feist_2019}. On the other
hand there is a trade-off between safety and coding flexibility. One of the most
common arguments against strict static checks is that they require producing a
lot of redundant code at little benefit in return, slowing down the development.

While this discussion may be very heated for certain fields of computer science,
the blockchain topic is a very specific case. As said before, smart contracts
are very vulnerable to bugs for the reason of being impossible to fix after
they are deployed. This effectively shifts the priorities towards having
more reliable code in the very first place, at the possible costs of slower
development. Indeed, losses caused by an error in a contract managing tokens
worth millions of dollars can be much more expensive than months of work
invested in the verification and testing.

There was research on the causes of the most common bugs in smart contracts in
the Ethereum network \cite{groce2020actual}. According to it, over 45\% of bugs
made in significant smart contracts have been in fact consequences of either
improper (or non-existent) data validation or holes in the access control. About
a fourth of those are marked as ``highly severe'', over a half of which are
found to be ``easy to exploit''. More generally, these issues involve relying on
assumptions that do not actually hold. Therefore, it is worth trying to look for
a way of making these assumptions more explicit and forcing the programmer to
take care of them according to the declared or inferred requirements.

In this chapter \emph{liquid types} are introduced as a remedy for the most
common problems of smart contracts. The aim is to enhance the existing type
system of Sophia with a dedicated context-dependent verification engine that
shall relate not only to the overall domains of data, but also express more
arbitrary logical predicates over it. That means the programmer shall be able to
put assumptions on the values of certain variables by attaching boolean expressions
to their types. More than that, the system shall automatically verify if the
requested properties hold, and provide its own assertions on the internal
functions provided by the language.

\section{History and overview}

The concept of liquid types comes from a broader term of \emph{refinement types}
which was introduced in 1991 by Tim Freeman \cite{Freeman91refinementtypes}. The
original idea defines a refined type as a base type enhanced with a predicate
that must hold for every inhabiting value. For example, a type of positive
integers can be described as $\{\nu : int | \nu > 0\}$ which is read as ``a type
of such integers $\nu$, that $\nu > 0$''. This allows putting more precise
restrictions on values and thus gives more flexibility in making invalid states
irrepresentable.

The name ``liquid type'' is an abbreviation for ``logically qualified data
type'' and originates in a 2008 paper by Patrick Rondon \cite{liquid} which
later became a part of his doctoral dissertation \cite{liquid_phd}. Liquid types
are a special case of a refined type system along with an inference algorithm.
That work serves as the main inspiration for this research, and the
implementation of Hagia is strictly based on the algorithms presented there.
From now on, the terms ``liquid type'' and ``refined type'' shall be used
interchangeably.

\section{Definition}
\label{liquid_types_definition}

Let $\Gamma$ denote the environment from the standard Hindley-Milner type
inference algorithm \cite{hindley, milner}. For the purpose of inference of
liquid types the \emph{liquid type environment} $\Gamma^*$ is introduced. The
difference between these two is that in $\Gamma^*$ each variable has assigned a
liquid type instead of a base type, and it keeps an additional \emph{path
  predicate} which is a conjunction of boolean expressions describing
assumptions derived from the context.

The following two definitions are reformulations of those from Rondon's paper,
being slightly adjusted to match the types in Sophia.

\begin{defi}
  A \underline{liquid type} is a dependent type that consists of
  \begin{itemize}
  \item A \underline{base type} $\mathbb{B}$
  \item A \underline{value handle} $\nu : \mathbb{B}$
  \item A \underline{predicate} $\mathbb{P}$ which is a conjunction of boolean
    expressions called \underline{qualifiers}
  \end{itemize}
  An expression $e$ has type $\{\nu : \mathbb{B} | \mathbb{P}\}$ in the liquid
  environment $\Gamma^*$ when $\Gamma \vdash e : \mathbb{B}$ and
  $\mathbb{P}[\nu/e]$ holds in $\Gamma^*$.
\end{defi}

Additionally, there is a restriction on $\mathbb{B}$ to be simple enough to let
$\mathbb{P}$ express reasonable properties of it. In Hagia it is limited to be
representable by either a primitive base type, such as \texttt{int} or
\texttt{bool}, or a type variable. Handling complex types, such as lists or
records, is described below.

\begin{defi}
  A \underline{liquid function type} consists of
  \begin{itemize}
  \item A list of \underline{liquid arguments} of form $\{x : \mathbb{T}\}$
    where $x$ is a variable and $\mathbb{T}$ is a liquid type
  \item A \underline{return type} $r$ where $r$ is a liquid type defined in an
    environment extended with assumptions derived from the arguments
  \end{itemize}
  It describes a function type, return type of which depends on the values of
  the arguments it has been applied to. In this notation, a liquid argument $\{x
  : \{x : \tau | \rho\}\}$ can be abbreviated to $\{x : \tau | \rho\}$. In the
  current implementation, arguments cannot depend on each other. See the
  \autoref{dependent_products} for the discussion.
\end{defi}

The presented types can be seen as a subset of $\Sigma$ and $\Pi$ types
respectively from the Martin--Löf dependent type theory
\cite{Martin-Loef_anintuitionistic}. The main difference are the restrictions on
their right-hand side, as liquid types allow a much tighter space of expressions
to be put there. For instance, a proper $\Pi$ type may perform a conditional
check on its argument and arbitrarily determine the return type according to it,
which the liquid functions considered here are not capable of.

Since Sophia is more complex than the simplified ML analysed in Rondon's work,
the following additional constructions are introduced:

\begin{defi}
  A \underline{liquid list type} consists of
  \begin{itemize}
  \item An \underline{element type} $\mathbb{T}$
  \item A \underline{length value handle} $\nu : \texttt{int}$
  \item A \underline{length predicate} $\mathbb{P}$ which is a conjunction of
    boolean expressions
  \end{itemize}
  The list $l$ is of type $\{\nu : \texttt{list}(\tau) | \mathbb{P}\}$ in
  $\Gamma^*$ when all of its elements are of type $\tau$ in $\Gamma^*$ and
  $\Gamma^* \vdash \mathbb{P}[\nu/\texttt{length}(l)]$
\end{defi}

In Hagia liquid lists may be referred as if they were integers. In such a case
they shall be cast to their lengths. For instance, the type $\{x : \texttt{int} |
  x > [2,1,3,7]\}$ can be interpreted as $\{x : \texttt{int} | x > 4\}$.

\begin{defi}
  A \underline{liquid record type} consists of
  \begin{itemize}
  \item A \underline{base type} $\mathbb{B}$ which is an identifier of a Sophia
    record type
  \item A list of \underline{liquid fields} of form $f : \mathbb{T}$ where
    $f$ is the name of a field of the record $\mathbb{B}$ and the base type of
    $\mathbb{T}$ is the type of that field.
  \end{itemize}

  We conclude that $\Gamma^* \vdash r : \{\mathbb{B} <: f_1 : \tau_1, f_2 :
  \tau_2\}$ when $\Gamma \vdash r : \mathbb{B}$ and each field $f_i$ has a
  respective type $\tau_i$ in $\Gamma^*$.
\end{defi}


\begin{defi}
  A \underline{liquid variant type} consists of
  \begin{itemize}
  \item A \underline{base type} $\mathbb{B}$ which is an identifier of a Sophia
    variant type
  \item A list of \underline{liquid constructors} of form $C(args)$ where $C$
    is the name of a constructor of $\mathbb{B}$ and $args$ is a list of liquid
    types applicable to that constructor.
  \end{itemize}

  We conclude that $\Gamma^* \vdash x : \{\mathbb{B} <: C_1(a_{1_1}, a_{1_2}) |
  C_2(a_{2_1}, a_{2_2})\}$ when $\Gamma \vdash x : \mathbb{B}$ and $x$ is
  defined by any of the constructors $C_i$ applied to respective parameters of
  types $a_{i_j}$. Note that not all constructors of $\mathbb{B}$ need to appear
  in the list. If that is the case, the liquid variant type is limited only to
  the values created with the mentioned constructors.
\end{defi}

Aside from the above, regular tuple and map constructions are allowed. They
cannot be equipped with logical qualifiers and disallow mutually dependent
refinements of the underlying types --- see the \autoref{dependent_products} for
the discussion.

\section{Syntax}

The type refinement system can work fine without any additional syntax. As long
as the existing constructions induce logical qualifications automatically, a
certain amount of safety is already provided. For example, division by zero is
prevented as the system pre-defines the type for the \texttt{/} operator to be
$(\{n : \texttt{int}\}, \{d : \texttt{int} | d \neq 0\}) \to \{r : \texttt{int}
| r = n / d\}$ which enforces the right operand not to be equal to zero
(\autoref{ssec:constr_expr} explains how it is constructed in more details).

However, giving the programmer the ability of setting their own restrictions
greatly extends the range of the properties that can be subjects of
verification. As shown in the \autoref{initial_assignment}, the space of
inferrable judgements is limited to some pre-defined set of expressions which
may not always be as precise as needed. An example where this makes a
significant difference can be found in the \autoref{double_token_storage}.
Therefore, a new syntax is provided which follows the definitions from the
\autoref{liquid_types_definition}. For the BNF grammar please refer to the
appendix, \autoref{liquid_types_syntax}.

\subsubsection{Examples}

\begin{itemize}
\item A positive integer: \texttt{\{x : int | x > 0\}}.
\item A non-empty list of odd integers: \texttt{\{l : list(\{e : int | e mod 2
    == 1\}) | l > 0\}}.
\item A function that computes maximum of two integers: \texttt{(\{a : int\},
    \{b : int\}) => \{r : int | r >= a \&\& r >= b\}}
\item A record \texttt{r = \{x : int\}} with \texttt{x} not equal to zero:
  \texttt{\{r <: x : \{self : int | self != 0\}\}}.
\item An option type that cannot be \texttt{None} and where \texttt{Some} wraps
  a positive integer: \texttt{\{option(int) <: Some(\{x : int | x > 0\})\}}.
\end{itemize}

\autoref{example_factorial_liquid} shows an example of a fully defined factorial
function which utilises liquid types. Note that in contrary to the previous
implementations it does not consider the case of a negative arugment, as the
type signature effectively forbids it.

\begin{code}[h]{lexers/sophia.py:SophiaLexer -x}{Factorial utilising liquid
    types}{example_factorial_liquid}
function
  factorial : {n : int | n >= 0} => {r : int | r >= 1 && r >= n}
  factorial(0) = 1
  factorial(x) = x * factorial(x - 1)
\end{code}

\section{Application}

The proposed type refinement system can secure many assumptions, which can be
used in verification of numerous algorithms. Of out all, the integer boundary
checking seems to be one of the most important. Thanks to the list length
predicates, it is possible to statically verify if certain access at point is
safe (that is, the index is not negative and is lesser than the length of the
list). Next, it can be used to prevent spends that are either negative or not
affordable by the contract. In a more practical view, such qualifications may be
used in a decentralised exchange with a registry of balances represented as
non-negative integers. These benefits are shown by examples in the
\autoref{outcome}.

Yet another feature that arises from the liquid types is the ability to assert
whether some point in code is reachable or not. Since $\Gamma^*$ stores a path
predicate, one can put restrictions or requirements on its satisfiability. This
can serve for eliminating dead code and preventing certain situations, like
failing pattern matching. An example for that is shown in the \autoref{outcome},
\autoref{similar_character_example}. The benefit is not only the increased
stability of the execution, but also extended flexibility in access control.
