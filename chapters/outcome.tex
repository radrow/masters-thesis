\chapter{Outcome}
\label{outcome}

\section{Summary}

The presented type refinement system introduces a framework for implementing
liquid types for Sophia. It covers most of the language's features focusing on
the data processing and arithmetics, providing also support for some
blockchain-specific functionalities, such as state management and contract
balance tracking. If needed, the algorithm can be extended to handle a greater
range of expressions and judgements, taking advantage of the fact that most of
the potentially desired cases can be reduced to the already implemented ones.
Hagia therefore serves as a working proof-of-concept and an inspiration for
utilising advanced type systems in smart contract development.

\section{Value and use cases}

In this section several examples of use of liquid types are presented. They show
how Hagia eliminates arithmetic bugs, improper validation, implicit crashes and
mistakes in logic. The example contracts are simplified to focus on the actual
problems, but one could easily imagine them participating in more advanced
decentralised systems. The error messages coming from the refiner have been
edited for readability, although the predicates preserve their meanings.

\subsection{Example: list length}

Liquid types have found their application by finding bugs in functions that had
been written as positive test cases for the project. These bugs most often
turned out to be oversights, which had been entirely ignored by the original
type checker. The following example presents a case of such a mistake. Let us
consider the following attempt to define a function computing the length of a
given list (\autoref{list_length_bug}).

\begin{code}[H]{lexers/sophia.py:SophiaLexer -x}{List length --- with a bug}{list_length_bug}
function
  len : {lst : list('a)} => {res : int | res == lst}
  len(lst) = switch(lst)
    []   => 0
    _::t => len(t)
\end{code}

If processed by Hagia--Sophia, the compiler yields the error:

\begin{Verbatim}[samepage=true]
Could not prove the promise at list_length.aes 2:3
arising from the assumption of triggering the 2nd branch of `switch`:
  res == lst
from the assumption
  res >= 0 && res == lst - 1
\end{Verbatim}

While it is indeed right that the result is greater or equal to zero, in the
second \texttt{switch} case the inferred assumption is that it is also lesser
than the length of \texttt{lst} by 1, instead of being equal to it. This not
only indicates that something has gone wrong, but also suggests what is the
issue. In this case, the bug was the missing incrementation of the tail's
length, so the following code compiles fine (\autoref{list_length_fixed}).

\begin{code}[H]{lexers/sophia.py:SophiaLexer -x}{List length --- fixed}{list_length_fixed}
function
  len : {lst : list('a)} => {res : int | res == lst}
  len(lst) = switch(lst)
    []   => 0
    _::t => len(t) + 1
\end{code}

\subsection{Example: similar character oversight}
\label{similar_character_example}

This is not a very common kind of bug thanks to syntax highlighting featured by
most of IDEs, but it can happen in some setups, especially if the code is copied
from external sources. Let us consider a factorial function implemented for
positive integers, as shown in the \autoref{faulty_factorial}.

\begin{code}[H]{lexers/sophia.py:SophiaLexer -x}{Factorial --- with a bug}{faulty_factorial}
function
  factorial : {x : int | x > 0} => int
  factorial(l) = 1
  factorial(n) = n * factorial(n - 1)
\end{code}

The code may seem fine at first glance, especially considering the simplicity of
the function and its rather casual definition. However, if evaluated by Hagia
the following error message is shown:

\begin{Verbatim}[samepage=true]
Found dead code at 4:2 by proving that
  !true && x == n && x > 0
never holds.
\end{Verbatim}

It turns out that the last clause of \texttt{factorial} is never going to be
visited. The pattern \texttt{n} is a variable, so it always matches. Hence, for
the error to be right, the pattern \texttt{l} must catch the argument regardless
of its value. And this is indeed true, because it is not an integer literal as
expected, but a variable ``\textit{l}'' instead, which in monospace fonts
frequently looks almost identical to the digit ``1''. The fixed version of the
contract in the \autoref{fixed_factorial} compiles fine.

\begin{code}[H]{lexers/sophia.py:SophiaLexer -x}{Factorial --- fixed}{fixed_factorial}
function
  factorial : {x : int | x > 0} => int
  factorial(1) = 1
  factorial(n) = n * factorial(n - 1)
\end{code}

\subsection{Example: spend-splitting contract}

For a more blockchain-specific example, a simple spend-splitting contract can be
considered (\autoref{split_spend}).

\begin{code}[H]{lexers/sophia.py:SophiaLexer -x}{Spend splitting contract --- no data
    validation}{split_spend}
include "List.aes"

contract SpendSplit =

  payable stateful entrypoint split(targets : list(address)) =
    let value_per_person = Call.value / List.length(targets)
    spend_to_all(value_per_person, targets)

  stateful function
    spend_to_all : (int, list(address)) => unit
    spend_to_all(_, []) = ()
    spend_to_all(value, addr::rest) =
      Chain.spend(addr, value)
      spend_to_all(value, rest)
\end{code}

The contract exposes an entrypoint which takes a list of addresses, splits the
call value by the number of receivers and sends each of them an even amount of
tokens (keeping the rest for itself). There are, however, several issues found
by the refiner. An attempt to compile it results in the following
errors:

\begin{Verbatim}[samepage=true]
Could not prove the promise at spend_split.aes 13:7:
  value =< $init_balance && value >= 0
from the assumption
  true
\end{Verbatim}
\begin{Verbatim}[samepage=true]
Could not prove the promise at spend_split.aes 13:7
arising from an application of "C.spend_to_all" to its 2nd argument:
  $balance_38 >= 0
from the assumption
  true
\end{Verbatim}
\begin{Verbatim}[samepage=true]
Could not prove the promise at spend_split 6:39
arising from an application of "(/)" to its right argument:
  n_105 != 0
from the assumption
  true
\end{Verbatim}

In other words:

\begin{enumerate}
\item In the 13th line the function does not check if the contract can afford
  the spend.
\item The next failure in the same line is caused by the broken promise of the
  balance being always non-negative. This is a direct cause of the previous
  error, as \texttt{Chain.spend} reduces the current balance by a given amount.
\item An insufficient-validation bug is present in the line 6, where the
  received amount is divided by the length of the list. Since the list is
  allowed to be empty, there is a case where this function crashes with a
  division-by-zero error.
\end{enumerate}

In contrary to the previous example, this contract directly deals with the token
flow, making it more blockhcain-specific. In order to fix it, a few additional
checks and assertions are to be added (\autoref{fixed_split_spend}). The
improved implementation differs from the previous one by an addtional
\texttt{require} statement and a more precise type declaration. Thus, with just
a little tweaking, the code is fixed and is accepted by the refiner\footnote{The
  given example is a preview for the intended functionality of the refiner, and
  it is not entirely supported yet. However, it is a part of the presented
  theory and can be implemented with some modifications of handling the
  functions' arguments. For a code snippet working with the current state of the
  project, please see the appendix, \autoref{fixed_spend_split}}.

\begin{code}[H]{lexers/sophia.py:SophiaLexer -x}{Spend splitting contract ---
    fixed}{fixed_split_spend}
include "List.aes"

contract SpendSplit =

  payable stateful entrypoint split(targets : list(address)) =
    require(targets != [], "NO_TARGETS")
    let value_per_person = Call.value / List.length(targets)
    spend_to_all(value_per_person, targets)

  stateful function
    spend_to_all : ( {v : int | v >= 0 && v * l =< Contract.balance}
                   , {l : list(address)}
                   ) => unit
    spend_to_all(_, []) = ()
    spend_to_all(value, addr::rest) =
      Chain.spend(addr, value)
      spend_to_all(value, rest)
\end{code}

\subsection{Example: double token storage contract}
\label{double_token_storage}

Previous examples show situations where liquid types prevent uncontrolled
crashing or arithmetic errors. Although the bugs can make trouble indirectly if
used as parts of some larger systems, they themselves do not expose anybody to
loses. To cover this case, this example visualises a scenario where one of the
peers can actually be a victim of a token theft, due to a ``small'' oversight.

The presented in the \autoref{double_token_storage_invalid} contract implements a
simple token storage which can be used by two hard-coded accounts. Their
balances are kept in \texttt{state} represented as a pair of integers. Users can
withdraw and store their tokens at any time with the \texttt{withdraw} and
\texttt{store} entrypoints. This example is quite simple, although in a
real-life scenario its extension could be a part of a staking contract used in
the \emph{hyperchains} protocol \cite{hyperchains}, for instance.

\begin{code}[H]{lexers/sophia.py:SophiaLexer -x}{Double token storage contract --- missing
    validation}{double_token_storage_invalid}
contract DoubleStore =
  type state = int * int
  entrypoint init() = (0, 0)

  stateful entrypoint withdraw(amount : int) =
    let (s1, s2) = state
    if(amount =< s1 && Call.caller == alice_address())
      put((s1 - amount, s2))
      Chain.spend(Call.caller, amount)
    elif(amount =< s1 && Call.caller == bob_address())
      put((s1, s2 - amount))
      Chain.spend(Call.caller, amount)
    else
      abort("INVALID_WITHDRAW")

  stateful entrypoint store() =
    let (s1, s2) = state
    if(Call.caller == alice_address())
      put((s1 + amount, s2))
    elif(Call.caller == bob_address())
      put((s1, s2 + amount))
    else
      abort("INVALID_SOTRE")

  function alice_address() =
    ak_2nqfyixM9K5oKApAroETHEV4rxTTCAhAJvjYcUV3PF1326LUsx
  function bob_address() =
    ak_2CvLDbcJaw3CkyderTCgahzb6LpibPYgeodmCGcje8WuV5kiXR
\end{code}

The contract compiles fine with the original type checker. As expected,
equipping Hagia exposes missing data validation by throwing the following
errors:

\begin{Verbatim}[samepage=true]
Could not prove the promise at double_store.aes 9:7:
  n_92 =< Contract.balance_0 && n_92 >= 0
from the assumption
  amount == n_92 && amount =< s1 && Call.caller == ak_2nqf...
\end{Verbatim}
\begin{Verbatim}[samepage=true]
Could not prove the promise at double_store.aes 12:7:
  n_179 =< Contract.balance_0 && n_179 >= 0
from the assumption
  amount == n_179 && amount =< s1 && Call.caller == ak_2CvL...
\end{Verbatim}

Indeed, the \texttt{withdraw} function does not check if the requested amount is
actually stored in the contract, nor whether it is non-negative. A proper
assertion makes the contract pass the liquid type check, as shown in the
\autoref{double_token_storage_passing}. From now on, the examples shall skip the
\texttt{store} entrypoint and the definitions of the hard-coded addresses.

\begin{code}[H]{lexers/sophia.py:SophiaLexer -x}{Double token storage contract --- addressing
    first Hagia errors}{double_token_storage_passing}
contract DoubleStore =
  type state = int * int
  entrypoint init() = (0, 0)

  stateful entrypoint withdraw(amount : int) =
    require(amount > 0 && amount =< Contract.balance, "INVALID_AMOUNT")
    let (s1, s2) = state
    if(amount =< s1 && Call.caller == alice_address())
      put((s1 - amount, s2))
      Chain.spend(Call.caller, amount)
    elif(amount =< s1 && Call.caller == bob_address())
      put((s1, s2 - amount))
      Chain.spend(Call.caller, amount)
    else
      abort("INVALID_WITHDRAW")
\end{code}

This contract is now accepted by the refiner, although its code does not make
any explicit use of liquid types. Since Hagia comes with syntax for defining
custom refinements, it is worth utilising them if possible. One of the most
obvious moves is to represent balances as non-negative numbers as shown in the
\autoref{double_token_storage_liquid_fail}.

\begin{code}[H]{lexers/sophia.py:SophiaLexer -x}{Double token storage contract --- putting
    explicit qualifications errors}{double_token_storage_liquid_fail}
contract DoubleStore =
  // state: alice's balance and bob's balance
  type state = {s1 : int | s1 >= 0} * {s2 : int | s2 >= 0}

  stateful
    entrypoint withdraw : {amount : int | amount > 0} => unit
    entrypoint withdraw(amount) =
      let (s1, s2) = state
        
      if(amount =< s1 && Call.caller == alice_address())
        put((s1 - amount, s2))
        Chain.spend(Call.caller, amount)
      elif(amount =< s1 && Call.caller == bob_address())
        put((s1, s2 - amount))
        Chain.spend(Call.caller, amount)
      else
        abort("INVALID_WITHDRAW")
\end{code}

After the changes, the contract no longer compiles. The refiner throws an error
and reports the following issue:

\begin{Verbatim}[samepage=true]
Could not prove the promise at double_store.aes 14:7 :
  s2_new >= 0
from the assumption
  s2_new == s2_old - amount && amount =< s1_old &&
  s1_old >= 0 && s2_old >= 0
\end{Verbatim}

The error message points out that the broken promise has been created in the
twelfth line, which is where \texttt{put} is used. It suggests that a possibly
out-of-domain value is assigned to the contract's state. For that to be the
case, a negative value must be assigned to one of the variables representing the
balances of the clients.

The variable \texttt{s2\_1} represents the updated balance of Bob. The refiner
fails to prove its non-negativity from a set of assumptions that suspiciously
assert that the withdrawn amount is not greater than \texttt{s1\_0}, which is
the initial balance of Alice. The bug is therefore in the line 11, specifically
in the \texttt{amount =< s1} check, which was supposed to verify if
\texttt{amount} is lesser or equal to \texttt{s2} instead. This is a very common
kind of mistake that usually comes from a bad habit of copying similar chunks of
code with the intent of applying synchronous modifications to them.

This bug is actually very dangerous. One of the users can withdraw more tokens
than they have actually stored in the contract. Therefore, if Alice puts 100AE
in it, Bob can freely take it all, leaving Alice with no way of retrieving her
funds.

Liquid types have therefore successfully prevented a faulty and easily
exploitable contract from being deployed. To fix it, the check has to be changed
to verify the right value as it is done in the
\autoref{double_token_storage_fixed}.

\begin{code}[H]{lexers/sophia.py:SophiaLexer -x}{Double token storage contract ---
    fixed}{double_token_storage_fixed}
contract DoubleStore =
  type state = {x1 : int | x1 >= 0} * {x2 : int | x2 >= 0}
  entrypoint init() = (0, 0)

  stateful entrypoint withdraw(amount : int) =
    require(amount > 0 && amount =< Contract.balance, "INVALID_AMOUNT")
    let (s1, s2) = state
    if(amount =< s1 && Call.caller == alice_address())
      put((s1 - amount, s2))
      Chain.spend(Call.caller, amount)
    elif(amount =< s2 && Call.caller == bob_address()) // Fixed
      put((s1, s2 - amount))
      Chain.spend(Call.caller, amount)
    else
      abort("INVALID_WITHDRAW")
\end{code}

The solution was rather simple and did not require complicated refinements to
reveal the bug. Although it was sufficient just to assert non-negative balances
in this case, another important assumption is that both balances sum up to the
balance of the whole contract. This could catch a hypothetical oversight where
the withdrawn funds are not subtracted at all, for instance. In order to define
it, the dependent products discussed in \autoref{dependent_products} have to be
implemented, and the final state declaration would look like in the
\autoref{double_token_storage_final}.


\begin{code}[H]{lexers/sophia.py:SophiaLexer -x}{Double token storage contract ---
    final}{double_token_storage_final}
contract DoubleStore =
  type state = {x1 : int | x1 >= 0} *
               {x2 : int | x2 >= 0 && x1 + x2 == Contract.balance}
  entrypoint init() =
    require(Call.value == 0, "NON_ZERO_INIT_VALUE")
    (0, 0)
\end{code}
