\chapter{Conclusions}

Liquid types have shown to be a valuable extension to the type system of Sophia.
The automated inference of facts makes them an exceptionally convenient
verification tool, as they require a little additional effort from the
programmer. More than that, the modifications of the code they enforce are often
needed anyway --- for example data validation in entrypoints is a mandatory part
regardless of the liquid types\footnote{On the other hand, the qualifications
  could be used to autogenerate validation of entrypoints implicitly. See the
  discussion, \autoref{making_use_of_quals}}.

The provided expressiveness is very often enough to catch most common errors. As
shown in the \autoref{outcome}, restricting integers to fit in certain
boundaries effectively prevents crashes on arithmetics and data queries, while
putting static checks on balances serves for the token flow control very well.
The proposed solution for pattern matching greatly helps to find unreachable
code and to prevent implicit crashing points.

Since liquid types contribute to the syntax, they improve the overall appearance
of contracts. When a reviewer or a potential client inspects the source, they
can relate to the assertions provided, making the program more trustworthy, as
logical qualifications can be utilised to describe many invariants. This not
only aids the reasoning about the code and helps debugging, but also brings a
marketing value by incentivising clients to use more verified products.
