\chapter{Discussion}

\section{Future work}

The implementation of liquid types for Sophia has not been entirely finished
yet. There are still some features of the language that are not covered or could
have been handled more appropriately. Aside from that, there is a range of
possible improvements that could enhance the performance and general
functionality of the refiner. This section provides a selection of problems and
ideas that may be addressed during future works on Hagia.

\subsection{Missing subtyping between empty product types}
\label{empty_tuples}

One of the yet unsolved issues is that empty product types can be described in
various ways, such that none of them is treated as equivalent to any of the
others. This happens when false qualifications are attached to different
projections, for example when $t_1 = \texttt{int} \times \{\nu : \texttt{int} |
\bot\}$ and $t_2 = \{\nu : \texttt{int} | \bot\} \times \texttt{int}$.
Practically, $t_1$ and $t_2$ are the same (empty), but this will not be solved
by Hagia due to constraint splitting which reduces $t_1 <: t_2$ to $\{\nu :
\texttt{int} | \bot\} <: \texttt{int}$ and $\texttt{int} <: \{\nu : \texttt{int}
| \bot\}$. The second subtyping does not hold, making the whole check fail.

A solution for that would be not spliting product types at all (including
domains of higher-arity functions) and validating product subtypes as a whole
instead. That would work, because if a false projection is assumed, then
regardless of its index every subtyping holds. Moreover, this approach
cooperates well with the liquid products proposed in the
\autoref{dependent_products}, as it considers mutual dependencies.

\subsection{Missing subtyping between effectively same variant types}
\label{effective_variant_types}

It happens that two variants with same sets of inhabitants are not treated as
equivalent, but instead there is a one-way subtyping relation between them. It
occurs when one type has a banished constructor and the other has it allowed,
but it cannot be used due to unsatisfiable qualifications of the parameters. For
example, the following relation holds:

$$\{\texttt{option(int)} <: \texttt{None}\} <: \{\texttt{option(int)} <:
\texttt{None} | \texttt{Some}(\{\nu : \texttt{int} | \bot\})\}$$

Unfortunately it does not work the other way around, despite the fact
that both types actually describe the same space of values, which in this case
is just a single \texttt{None}. The cause of this issue is that tag predicates
are entirely detached from the context of constructors' arguments, and the
system does not see that it is actually impossible to create a value of the
second type using \texttt{Some} constructor. Thus, after splitting the reversed
case, the refiner fails to satisfy the tag predicate constraint

$$\Gamma^* \vdash\{\nu : \texttt{option(int)} | \top\} <: \{\nu :
\texttt{option(int)} | \lnot(\exists x. \nu = \texttt{Some}(\_))\}$$

A possible solution for that would be to enhance tag predicates with additional
information about the parameters, so the refiner shall infer that \texttt{Some}
is excluded from the domains of both types. However, if the problem introduced
in the \autoref{empty_tuples} is addressed, a better way to do it emerges. Since
after the fix all projections are tied, tag predicates could be replaced with
dummy arguments of the constructors. They would be of liquid \texttt{unit} type
with a qualification stating whether the respective constructor is triggered.
More than that, after implementing liquid products from the
\autoref{dependent_products} it would be possible to assert that only one
constructor can be used to build a single value. Going back to the previous
example, the subtyping would be rewritten to

\begin{align*}
  & \{\texttt{option(int)} <: \texttt{None}(\{\texttt{is\_none} : \texttt{unit} | \rho_1\}) | \texttt{Some}(\{\texttt{is\_some} : \texttt{unit} | \bot\}, \texttt{int})\} <:\\
  & \{\texttt{option(int)} <: \texttt{None}(\{\texttt{is\_none} : \texttt{unit} | \rho_2\}) | \texttt{Some}(\{\texttt{is\_some} : \texttt{unit} | \rho_3\}, \{\nu : \texttt{int} | \bot\})\}
\end{align*}

where each $\rho_i$ is a liquid template. Note that in this setup it is no longer
needed to falsify parameters of banished constructors, because they are
considered in contradictory contexts anyway. With that setup, the subtyping
should work in both directions as expected.

\subsection{Incremental solving}

Z3 builds the solving environment incrementally. That means it keeps its state
between queries, which is reused for handling the subsequent assertions. If
properly utilised, it can have a great impact on the performance. One of the
ways of interacting with this feature are the \texttt{push} and \texttt{pop}
instructions, which allow making snapshots of the current state and returning to
it afterwards.

Currently, Hagia makes little use of it; for instance, it always cleans up the
whole context and redefines the solving enviroment from scratch with every
constraint. This behaviour could be changed to respect the dependencies between
environments. Except from some initial cases, almost every constraint is created
in an environment that is a direct extension of another one, which could be
reflected in the interaction with the solver.

One way to realise it is to build up a tree structure where the constraints are
stored in the nodes and vertices describe updates of the environments. The
iterative weakening then traverses through the graph, sending new assertions to
the solver while moving to each child node. When returning to the parent, the
\texttt{pop} instruction is called to revert Z3 to a respective state.

For example, considering a simple function involving an \texttt{if} expression:

\begin{lstlisting}[language=sophia]
function trim_neg(a : int) : {res : int | res >= 0} =
  let b = a > 0
  if(b) a else 0
\end{lstlisting}

The following set of constraints is generated (simplified for readability):

\begin{align*}
  & \Gamma^*_t \vdash \{\nu : \texttt{int} | \nu = a\} <: \{\nu : \texttt{int} | \nu
    \geq 0\}\\
  & \Gamma^*_e \vdash \{\nu : \texttt{int} | \nu = 0\} <: \{\nu : \texttt{int} | \nu
    \geq 0\}
\end{align*}

where

\begin{align*}
  & \Gamma^* = a : \texttt{int}, b : \{\nu : bool | \nu = a > 0\}\\
  & \Gamma^*_t = \Gamma^*, b\\
  & \Gamma^*_e = \Gamma^*, \lnot b
\end{align*}

After the optimisation, the following graph is built:

\begin{adjustbox}{margin=0 30 0 30}
% https://tikzcd.yichuanshen.de/#N4Igdg9gJgpgziAXAbVABwnAlgFyxMJZAJgBpiBdUkANwEMAbAVxiRAB12BxOgW17oA9AFQB9GAAJONKHTgALKe2CcwTCYiU4YADxz7gWMDgC+EgD5K1EgLwSADJzMAeTZxXtrb9tr0GjphZW6pwA5jAAjg5OICak6Ji4+IQo9uRUtIwsbEo8-EJiOEoycoruqurevvo4hsZmlhW2EnROEq5KHl5aujV1gY2eIezhUfZKJrHxIBjYeAREAIykixn0zKyIIFMJc8lLpPZrWZvbJhkwUOEIKKAAZgBOELxIZCA4EEjLmRtsnNUGORwGAPUwACgARgBKEDUBh0CEwBgABUS8xSIAeWFC8hwOxAj2er2oHyQaRA8MRKLR+y2DBgdzx1HW2S2-16gLgwNBJjBnAYkCK0PxhJeiAAzCTPog3izTuy-LV6UUTBDmnQJAA+ByxCgmIA
\begin{tikzcd}[cramped]
  & {} \arrow[d, "\texttt{let }b = a > 0"]\\
  & \Gamma^* \arrow[ld, "\texttt{assert}(b)"'] \arrow[rd, "\texttt{assert}(\lnot b)"] &\\
  \Gamma^*_t \vdash \{\nu : \texttt{int} | \nu = a\} <: \{\nu : \texttt{int} | \nu
  \geq 0 \} & & \Gamma^*_e \vdash \{\nu : \texttt{int} | \nu = 0\} <: \{\nu :
  \texttt{int} | \nu \geq 0\}
\end{tikzcd}
\end{adjustbox}

The environment $\Gamma^*$ can now be reused for handling both \texttt{if}
branches. This prevents many definitions and predicates being defined twice in
the SMT, which saves a lot of the solving time. The overall complexity of the
iterative weakening is thus divided by the number of constraints, which shall
significantly affect the performance.

\subsection{Relationship analysis for the initial assignment}

The proposed solution for the initial assignment is rather naive as it assumes
relationships between all variables of the same type and combines them with all
possible operations to cover as many cases as possible. A similar strategy is
applied to the integer literals, which are picked very arbitrarly. It results in
many irrelevant or otherwise absurd qualifiers, leaving a quadratic number of
statements to be examined. Not only does it affect the memory, but more
importantly it reduces the performance of the iterative weakening.

Some analysis to make this process more sensible could be performed. For
instance, a newly introduced variable could relate only to the variables that
contributed to its definition. Similar heuristics could be applied to integers.

\subsection{Parallelisation}

Erlang is mostly known for its efficient parallelism through \emph{processes}.
Making use of them could bring a big improvement to the performance. A good
place to start is the AST preprocessing and constraint generation procedures,
because they work only with either read-only or append-only data.

More importantly, the iterative weakening, which is the most computationally
intensive part, can also take advantage of it. Independent processes could
weaken the constraints separately by removing the conflicting qualifiers on
their own and apply the changes to the assignment after each iteration. The
parallelism does not break the correctness of the algorithm --- processing more
constraints in a single round does not limit the progress in any way.

This optimisation requires a proper synchronisation. One of the strategies is to
equip all the wobbly qualifiers with indices, so they can be reported by the
working processes and removed by the supervisor after each iteration. This
requires altering the underlying representation of the assignments to use
\texttt{maps}, \texttt{proplists} or some other indexed data collection for
representation of the predicates.

The proposed strategy has not been tested, nor has any attempt of implementing
it been done in this project. Moreover, it might be challenging to combine it
with incremental solving. Although there are no guarantees, the idea is
promising and should be considered during the further development.

\subsection{Liquid types for termination checking}

Since the theory is capable of making judgements about values, it is able to
identify some cases of non-terminating function. With the size-change measure
\cite{10.1145/373243.360210, Sereni05terminationanalysis} method, some
additional subtyping contraints could be generated to ensure the convergence.
This idea has been already implemented in LiquidHaskell
\cite{vazou2014refinement} --- a refinement type checker for Haskell.

The size measure is trivially definable for most Sophia types, especially the
inductive ones. Integers could be cast to their absolute values or limited to
natural numbers with logical qualifications. This gives an idea for a slick
implementation of a simple totality checker, which could prevent some infinite
recursions and allow more flexibility in defining custom refinements.

\subsection{Liquid products with mutual dependencies}
\label{dependent_products}

In product types (such as tuples, records and some cases of variants) the
projections cannot be declared to depend on each other. Fixing it would
considerably increase the flexibility of qualifications on complex data. This,
for instance, would allow creating a type for well-formed integer ranges,
defined as tuples $(n, m) : int \times int$, where $n \leq m$. Another use-case
is keeping some additional information about the data, like a tuple of a list
and an integer describing its length for a constant-time access. It is worth
noting that higher-arity functions may benefit from that in a similar manner ---
in that case a linear dependency can be now achieved via currying, however.

\subsection{Making use of the qualifications during byte code generation}
\label{making_use_of_quals}

Liquid typing provides information that can be taken into account during the
final compilation phases. For instance, if a variable has been proven to contain
a fixed value, it can be safely inlined. Another example could be a situation
where it is clear which constructor has been used to create the value under a
given variant variable. This could reduce the number of unnecessary checks and
thus save gas.

Furthermore, the qualifications on contravariant types of entrypoints do not
need to be disallowed. Since they are valid Sophia expressions, they can be used
for automated generation of data validation. The downside of this is the lesser
transparency --- implicit behaviours might not look right to some programmers.
An option worth considering is to make it adjustable by the user with a flag or
some other configuration method.

\section{Alternative approaches}

While liquid types are a reliable way of dealing with the common issues of smart
contracts, there are some alternatives that appear promising as well. Static
verification of programs is a broad field of computer science, so the presented
solutions are only a selection of what can be taken into account.

\subsection{Dependent type theory}

As stated before, the proposed system can be seen as a simplified version of
Martin-Löf's dependent type theory. Utilisation of full-featured dependent types
in Sophia would drastically increase the range of expressable properties, as it
has little to no limitations on the qualifications being used and allows
building very arbitrary data types. For instance, one could wrap the type of
some expression with \texttt{option}, but only if the computation has a chance
to fail due to the valuation of the variables in the context. This is
inexpressible with liquid types, as it requires having a variable base type.

On the other hand there are some drawbacks that one needs to consider. First,
this theory is much more complex than liquid types. Since types can be results
of arbitrary computations, the type inference might become very hard, if not
undecidable, in some cases. Consequently, the programmer may be forced to
provide proofs for declared theorems on their own. Not only does this work tend
to be very tedious, but it also often requires a deep understanding of the
underlying theory.

Next thing is that with this theory it is not possible to rely on an SMT solver
entirely. The reason for it is that dependent types do not depend solely on
logical formulations, but rather on arbitrary data. Therefore, the system would
need to interpret the code on its own, which would have a negative impact on the
performance.

Needless to say, one must keep in mind that this field of computer science is
developing very dynamically and is attracting more interest over time. There
already exist some mature implementations of dependent types such as Coq
\cite{10.5555/993954}, Agda \cite{agda} and Idris \cite{idris}. The former is
mostly focused on theorem proving and has already found its application in smart
contract development \cite{10.1145/3437992.3439934}. The other two are
general-purpose programming languages, which makes them very promising for that
as well.

\subsection{Contract checking}

Contract\footnote{In that theory the term ``contract'' has a different meaning
  and is not related to smart contracts.} checking \cite{Bruce_contractsfor} is
yet another approach to the automated verification. This theory puts more focus
on expressions than types, but the purpose is rather similar to liquid types and
both solutions have comparable expressiveness. For a bigger picture, there is a
master's thesis \cite{wojciechbaranowski2014} that provides a detailed
comparison between Haskell Contracts Checker \cite{10.1145/2480359.2429121} and
LiquidHaskell \cite{vazou2014refinement}.

\subsection{Property-based testing}

Property-based testing stays somewhere in the middle between unit testing and
automated theorem proving. The idea is to let the programmer express some
logical properties of functions and values, which serve then as a base for
automatic test generation aiming to find counter-examples for the given
assumptions. One of the most famous implementations is QuickCheck
\cite{10.1145/357766.351266} which was originally created for verification of
Haskell programs, and has inspired similar tools (usually of the same name) for
many other programming languages.

The approach is very different though. Most importantly, tests do not guarantee
that the given assumptions necessarily hold, in contrast to the already shown
solutions. On the other hand, they are much easier to manage by the programmer;
they do not require actual derivation of proofs, which in many cases provides
enough security at lower cost of maintenance. Because of that difference, it is
worth considering using both automated testing and a strong type system
simultanously, as they do not conflict with each other.
